% !TEX program = xelatex

%This document had would had been used on Ways to Singularity, which is a website that supports MathJax. So some html elements may occur in this document. DELETE THIS when publishing. (If you are not very clear on the grammar I used here, read the academic publication called Time Traveller's Handbook of 1001 Tense Formations by Dr Dan Streetmentioner, which had would had been publish in year 220010 of Gregorian Calendar.)

\documentclass[12pt,a4paper]{book}


%%%%% Page settings  %%%%%
\addtolength{\textheight}{2.0cm}
\addtolength{\voffset}{-2cm}
\addtolength{\hoffset}{-1.0cm}
\addtolength{\textwidth}{2.0cm}

%\allowdisplaybreaks


%%%%%% Math %%%%%%%
\usepackage{amsthm,amsfonts,amssymb,bm}
\usepackage{mathrsfs}
\usepackage[fleqn]{amsmath}
\usepackage{subeqnarray}
\usepackage{tabulary}



%%%%% MISC  %%%%%%

%\usepackage{color}
\usepackage[usenames,dvipsnames]{color}
\usepackage{url}
\usepackage{ulem}
\usepackage{indentfirst}   % Indent first line of a paragraph
%\usepackage{textcomp}

\usepackage{enumerate}


%%%%%%   Here is the configuration for chinese. setmainfont is the default font of the text.
\usepackage[cm-default]{fontspec}
\usepackage{xunicode}
\usepackage{xltxtra}


%%%%%%% Figure, Diagram, Caption settings  %%%%%
%\usepackage{tikz}
%\usetikzlibrary{mindmap,trees}

\usepackage{graphicx}
%\usepackage{graphics}
%\usepackage[hang,small,bf]{caption}
%\setlength{\captionmargin}{50pt}


%\graphicspath{{Figures/}



%includeonly{}

\begin{document}
\title{TITLE}
\author{{\bf MA} Lei  \\
@ Interplanetary Immigration Agency \\
{\small\em \copyright \ Draft date \today}}
\date{}
%\begin{document}
%\maketitle


%%%%%% Redefine some math commands and environments. %%%%

\newcommand{\dd}{\mathrm d}

%\newcommand{\HH}{\mathcal H}

%\newcommand{\CN}{{\it Cosmologia Notebook}}

%\newenvironment{eqnset}
%{\begin{equation}\left \bracevert \begin{array}{l}}
%{\end{array} \right. \end{equation}}

%\newenvironment{eqn}
%{\begin{equation}\left \bracevert \begin{array}{l}}
%{\end{array} \right. \end{equation}}


%%%%%%%%%%%%%%%%%%%%%%%%%%%%%%%%%%%%%%%%%%%%%%%%%%%%%%%%
%%%%%%%%%%%%%%    Let's Start Typing     %%%%%%%%%%%%%%%
%%%%%%%%%%%%%%%%%%%%%%%%%%%%%%%%%%%%%%%%%%%%%%%%%%%%%%%%





%%%%%%%%  CHAHCA   %%%%%%%%%%
%\setcounter{chapter}{10}
%\chapter{Tensors and Local Symmetries}

\pagestyle{empty}

\renewcommand{\arraystretch}{1.5}

\begin{center}
Calculus on One Page
\end{center}



\begin{tabular}{c|c}\hline
	Differential & Integral \\ 
	\hline
	\multicolumn{2}{c}{Fundamental Theorem of Calculus} \\ \hline 
	 $\frac{\mathrm d}{\mathrm dx}\int_a^x f(t)\mathrm dt = f(x) $ & 
	$\int_a^x \frac{\mathrm d}{\mathrm dt}f(t)\mathrm dt = f(x)-f(a)$  \\   \hline

	\multicolumn{2}{c}{Rules} \\  \hline
	$[c_1 u(x)+c_2 v(x)]'=c_1 u'(x)+c_2 v'(x)$ &  $\int\left[c_1f(x) + c_2 g(x)\right]\mathrm dx = c_1\int f(x)\mathrm dx + c_2 \int g(x)\mathrm dx$   \\
	$(u v)' = u'v + u v'$  & $\int u' v \mathrm dx = u v - \int uv'\mathrm dx$  \\
	$\frac{\mathrm df(u(x))}{\mathrm dx} = \frac{\mathrm df}{\mathrm du}\frac{\mathrm du}{\mathrm dx}$    &   $\int \frac{\mathrm d f}{\mathrm d u}\frac{\mathrm du}{\mathrm dx} \mathrm dx= f(u(x))+c $  \\   \hline
	\multicolumn{2}{c}{Mean Value Theorem} \\	  \hline
	$\frac{F(b) - F(a)}{b-a} = F'(\xi)$  &  $\int_a^b f(x) \mathrm dx = (b-a)f(\xi)$ \\ \hline
	\multicolumn{2}{c}{Useful Equations} \\   \hline
	$(x^n)'=nx^{n-1}$   &  $\int x^n \mathrm dx = \frac{x^{n+1}}{n+1}+c $  \\
	$(\sin x)'=\cos x$   &   $\int \cos x \mathrm dx = \sin x +c$    \\
	$(\ln x)' = \frac{1}{x}$   &   $\int \frac{1}{x} \mathrm dx = \ln\vert x \vert +c$  \\
	$(e^x)'=e^x$  &   $\int e^x \mathrm dx = e^x +c$  \\   \hline
	\multicolumn{2}{c}{Multivariable Calculus (Fundamental Theorem)} \\  \hline
	\multicolumn{2}{c}{$\int_{\partial\Omega} \bm \omega  = \int_{\Omega} \mathbf d\bm \omega$ }  \\  \hline
\end{tabular}

\vspace{5ex}
{\bf  Notes:}


\begin{itemize}
\item
One can derive other useful equation using just these in this table.
\item
The three equations of fundamental theorem for multivariable calculus can be derived using the exterior derivative.
\end{itemize}















\end{document}